\documentclass{article}
\usepackage[utf8]{inputenc}
\usepackage{amssymb}

\begin{document}

    \section{Desenvolvimento da EDF}
    
        A equação diferencial dada pelo problema é:
        
        \begin{equation}
            \frac{d^2\phi(x)}{dx^2}  + B_g^2 \phi = 0.
        \end{equation}
        
        Aplicando a fórmula da diferença
        
        \begin{equation}
            f_{xx|i} = \frac{f_{i+1} - 2 f_i + f_{i-1}}{\Delta x^2} 
        \end{equation}
        
        na Eq. (1) temos que
        
        \begin{equation}
            \frac{\phi_{i+1} - 2 \phi_i + \phi_{i-1}}{\Delta x^2} + B_g^2 \phi_i = 0
        \end{equation}
        
        que resulta em
        
        \begin{equation}
            a \phi_{i+1} + b \phi_i + a \phi_{i-1} = 0
        \end{equation}
        
        onde
        
        \begin{equation}
            a = \frac{1}{\Delta x^2}
        \end{equation}
        
        e
        
        \begin{equation}
            b = -\frac{2}{\Delta x^2} + B_g^2
        \end{equation}
    
    \section{Prova da consistência}

        Multiplicando ambos lados da Eq. (3) por $\Delta x^2$ temos
    
        \begin{equation}
            \phi_{i+1} - 2 \phi_i + \phi_{i-1} + B_g^2 \Delta x^2 \phi_i = 0
        \end{equation}

        que pode ser reescrita como

        \begin{equation}
            ( \phi_{i+1} + \phi_{i-1} )  + (B_g^2 \Delta x^2  -  2) \phi_i = 0.
        \end{equation}

        Através da séria de Taylor pode se escrever

        \begin{equation}
            \phi_{i+1} = \phi_{i}  + \frac{d\phi}{dx} \Delta x  +  \frac{1}{2}\frac{d^2\phi}{dx^2} \Delta x^2  +  \frac{1}{6}\frac{d^3\phi}{dx^3} \Delta x^3 + \frac{1}{24}\frac{d^4\phi}{dx^4} \Delta x^4 + ...
        \end{equation}

        e

        \begin{equation}
            \phi_{i+1} = \phi_{i}  - \frac{d\phi}{dx} \Delta x  +  \frac{1}{2}\frac{d^2\phi}{dx^2} \Delta x^2  -  \frac{1}{6}\frac{d^3\phi}{dx^3} \Delta x^3 + \frac{1}{24}\frac{d^4\phi}{dx^4} \Delta x^4 - ...
        \end{equation}

        logo

        \begin{equation}
            \phi_{i+1} + \phi_{i-1}    =     2\phi_{i}   +  \frac{d^2\phi}{dx^2} \Delta x^2  + \frac{1}{12}\frac{d^4\phi}{dx^4} \Delta x^4 + ...
        \end{equation}
        
        Substituindo na EDF temos que

        \begin{equation}
            (2\phi_{i}   +  \frac{d^2\phi}{dx^2} \Delta x^2  + \frac{1}{12}\frac{d^4\phi}{dx^4} \Delta x^4 + ...) - (2 + \Delta x^2 B_g^2 ) = 0
        \end{equation}

        Dividindo por $\Delta x^2$ e rearranjando, temos que 

        \begin{equation}
            \frac{d^2 \phi}{dx^2}  - B_g \phi = \frac{d^2\phi}{dx^2} \Delta x^2  + \frac{1}{12}\frac{d^4\phi}{dx^4} \Delta x^4 + ...
        \end{equation}

        Como todos os termos do lado direito na Eq. acima estão multiplicados por $\Delta x$ elevada a alguma potencia positiva, podemos inferir que
        
        \begin{equation}
            \Delta x \rightarrow 0      \Rightarrow       \frac{d^2 \phi}{dx^2}  - B_g \phi   \rightarrow 0
        \end{equation}

        O que prova que a EDF é consistente.
        
\end{document}
