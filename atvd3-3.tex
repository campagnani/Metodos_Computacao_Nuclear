\documentclass{article}
\usepackage[utf8]{inputenc}
\usepackage{amssymb}

\begin{document}

    \section{Desenvolvimento da EDF}
        Equação da Difusão:
        
        \begin{equation}
            \triangledown^2 \phi (\vec r) - \frac{1}{L^2} \phi(\vec r) = - \frac{S(\vec r)}{D}
        \end{equation}
        
        Laplaciano para o caso unidimensional de coordenadas cilíndricas:
        
        \begin{equation}
            \triangledown =  \frac{\partial^2}{\partial r^2}  + \frac{2}{r} \frac{\partial}{\partial r}
        \end{equation}
        
        Pelas Eq. (1) e (2) temos que
        
        \begin{equation}
            \frac{\partial^2\phi}{\partial r^2}  + \frac{2}{r} \frac{\partial\phi}{\partial r} - \frac{1}{L^2} \phi = \frac{S}{D}
        \end{equation}

        e considerando que a fonte é linear, para $r \neq 0$ temos que $S = 0$, logo

        \begin{equation}
            \frac{\partial^2\phi}{\partial r^2}  + \frac{2}{r} \frac{\partial\phi}{\partial r} - \frac{1}{L^2} \phi = 0
        \end{equation}
        
        Aplicando as formulas da diferença
        
        \begin{equation}
            f_{x|i} = \frac{f_{i+1} - f_{i-1}}{2 \Delta x} 
        \end{equation}
        
        e
        
        \begin{equation}
            f_{xx|i} = \frac{f_{i+1} - 2 f_i + f_{i-1}}{\Delta x^2} 
        \end{equation}
        
        na Eq. (3) temos que
        
        \begin{equation}
            \frac{\phi_{i+1} - 2 \phi_i + \phi_{i-1}}{\Delta x^2} + \frac{\phi_{i+1} - \phi_{i-1}}{2 \Delta x} - \frac{1}{L^2} \phi_i = 0
        \end{equation}
        
        que resulta em
        
        \begin{equation}
            a \phi_{i+1} + b \phi_i + c \phi_{i-1} = 0
        \end{equation}
        
        onde
        
        \begin{equation}
            a = \frac{1}{\Delta x^2} + \frac{1}{2 \Delta x}
        \end{equation}
        
        e
        
        \begin{equation}
            b = -\frac{2}{\Delta x^2} - \frac{1}{L^2}
        \end{equation}
        
        e
        
        \begin{equation}
            c = \frac{1}{\Delta x^2} - \frac{1}{2 \Delta x}.
        \end{equation}






     \section{Prova da consistência}

        Reescrevendo a Eq. (8) por extenso, temos
    
        \begin{equation}
              (  \frac{1}{\Delta x^2} + \frac{1}{2 \Delta x} )      \phi_{i+1} 
            + ( -\frac{2}{\Delta x^2} - \frac{1}{L^2} )             \phi_i 
            + (  \frac{1}{\Delta x^2} - \frac{1}{2 \Delta x} )      \phi_{i-1} = 0
        \end{equation}

        Multiplicando ambos lados por $\Delta x^2$ temos

        \begin{equation}
              (  1 + \frac{\Delta x  }{2} )      \phi_{i+1} 
            + ( -2 - \frac{\Delta x^2}{L^2} )    \phi_i 
            + (  1 - \frac{\Delta x  }{2} )      \phi_{i-1} = 0
        \end{equation}

        Que pode ser reescrita como

        \begin{equation}
            ( m \phi_{i+1} + n \phi_{i-1} )  + ( -2 - \frac{\Delta x^2}{L^2} ) \phi_i = 0.
        \end{equation}

        onde
        
        \begin{equation}
            m = 1 + \frac{\Delta x  }{2}
        \end{equation}
        
        e
        
        \begin{equation}
            n = 1 - \frac{\Delta x  }{2}
        \end{equation}

        Através da séria de Taylor pode se escrever

        \begin{equation}
            \phi_{i+1} = \phi_{i}  + \frac{d\phi}{dx} \Delta x  +  \frac{1}{2}\frac{d^2\phi}{dx^2} \Delta x^2  +  \frac{1}{6}\frac{d^3\phi}{dx^3} \Delta x^3 + \frac{1}{24}\frac{d^4\phi}{dx^4} \Delta x^4 + ...
        \end{equation}

        e

        \begin{equation}
            \phi_{i-1} = \phi_{i}  - \frac{d\phi}{dx} \Delta x  +  \frac{1}{2}\frac{d^2\phi}{dx^2} \Delta x^2  -  \frac{1}{6}\frac{d^3\phi}{dx^3} \Delta x^3 + \frac{1}{24}\frac{d^4\phi}{dx^4} \Delta x^4 - ...
        \end{equation}

        logo

        \begin{equation}
            m \phi_{i+1} + n\phi_{i-1}      = (m+n)  \phi_{i}  
                                            + (m-n)               \frac{d\phi}  {dx  } \Delta x 
                                            + (m+o)  \frac{1}{2}  \frac{d^2\phi}{dx^2} \Delta x^2 
                                            + (m-n)  \frac{1}{6}  \frac{d^3\phi}{dx^3} \Delta x^3 
                                             ...
        \end{equation}%+ (m+o)  \frac{1}{24} \frac{d^4\phi}{dx^4} \Delta x^4 +

        Note que

        \begin{equation}
            m + n = 2
        \end{equation}

        enquanto que

        \begin{equation}
            m - n = \Delta x
        \end{equation}

        substituindo temos que

        \begin{equation}
            m \phi_{i+1} + n\phi_{i-1}      = 2  \phi_{i}  
                                            +              \frac{d\phi}  {dx^2} \Delta x^2 
                                            +              \frac{d^2\phi}{dx^2} \Delta x^2 
                                            + \frac{1}{6}  \frac{d^3\phi}{dx^3} \Delta x^4 
                                             ...
        \end{equation}%+ (m+o)  \frac{1}{24} \frac{d^4\phi}{dx^4} \Delta x^4 +
        
        Substituindo na EDF temos que

        \begin{equation}
            (
            2  \phi_{i}  
            +              \frac{d\phi}  {dx  } \Delta x^2 
            +              \frac{d^2\phi}{dx^2} \Delta x^2 
            + \frac{1}{6}  \frac{d^3\phi}{dx^3} \Delta x^4 
             ...
            ) 
            - 
            (2 + \frac{\Delta x^2}{L^2} ) = 0
        \end{equation}

        Dividindo por $\Delta x^2$ e depois de alguns algebrismos, temos que 

        \begin{equation}
            \frac{d^2\phi}{d r^2}  + \frac{2}{r} \frac{d\phi}{d r} - \frac{1}{L^2} \phi 
            =
             \frac{1}{6}  \frac{d^3\phi}{dx^3} \Delta x^2 
           + \frac{1}{12} \frac{d^4\phi}{dx^4} \Delta x^2 +
             ...
        \end{equation}



        Como todos os termos do lado direito na Eq. acima estão multiplicados por $\Delta x$ elevada a alguma potencia positiva, podemos inferir que
        
        \begin{equation}
            \Delta x \rightarrow 0      \Rightarrow       \frac{d^2\phi}{d r^2}  + \frac{2}{r} \frac{d\phi}{d r} - \frac{1}{L^2} \phi    \rightarrow 0
        \end{equation}

        O que prova que a EDF é consistente.
\end{document}
